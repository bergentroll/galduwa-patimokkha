\section{Uncertain Cases}
\label{unc}

\begin{center}
  Venerables, these two uncertain cases come up for recitation.
\end{center}

\pdfbookmark[3]{Uncertain Case 1}{unc1}
\subsection*{\hyperref[aniy1]{Uncertain Case 1: The first uncertain training precept}}
\label{unc1}
If any bhikkhu should sit down together with a woman, one [man] with one [woman], privately, on a concealed seat [that is] sufficiently fit for doing [it], [and then if] a female lay-follower whose words can be trusted having seen that, should speak according to one of three cases: according to disqualification, according to what concerns the community in the beginning and in the rest [of the procedure], or according to expiation, [then] the bhikkhu who is admitting the sitting down should be made to do [what is] according to one of three cases: according to disqualification, or according to what concerns the community in the beginning and in the rest [of the procedure], or according to expiation, or according to whatever that female lay-follower whose words can be trusted should say, according to that the bhikkhu is to be made to do. This is an uncertain case.

\pdfbookmark[3]{Uncertain Case 2}{unc2}
\subsection*{\hyperref[aniy2]{Uncertain Case 2: The second uncertain training precept}}
\label{unc2}
But even if the seat is neither concealed nor sufficiently fit for doing it, but is sufficient for speaking suggestively to a woman with depraved words: if any bhikkhu should sit down on such a seat together with a woman—one [man] with one [woman], privately—[and then if] a female lay-follower whose words can be trusted having seen that, should speak according to one of two cases: according to what concerns the community in the beginning and in the rest, or according to expiation, [then] the bhikkhu admitting the sitting down is to be made to do according to one of two cases: according to what concerns the community in the beginning and in the rest [of the procedure], or according to expiation, or according to whatever that female lay-follower whose words can be trusted should say, according to that the bhikkhu is to be made to do, this too is an uncertain case.

\medskip

\begin{center}
Venerables, the two uncertain cases have been recited.

\smallskip

Concerning that I ask the Venerables: [Are you] pure in this?\\
A second time again I ask: [Are you] pure in this?\\
A third time again I ask: [Are you] pure in this?

\smallskip

The venerables are pure in this, therefore there is silence, thus I bear this [in mind].
\end{center}

\begin{outro}
  The recitation of the uncertain [cases] is finished
\end{outro}

\clearpage
