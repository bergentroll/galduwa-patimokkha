\section{Nidān'uddeso}
\label{nidan'uddeso}

Suṇātu me bhante / āvuso saṅgho, ajj'uposatho paṇṇaraso / cātuddaso / sāmaggo, yadi saṅghassa pattakallaṁ, saṅgho uposathaṁ kareyya pātimokkhaṁ uddiseyya.

Kiṁ saṅghassa pubbakiccaṁ?

Pārisuddhiṁ āyasmanto ārocetha, pātimokkhaṁ uddisissāmi, taṁ sabbeva santā sādhukaṁ suṇoma manasi karoma. Yassa siyā āpatti, so āvikareyya, asantiyā āpattiyā tuṇhī bhavitabbaṁ, tuṇhībhāvena kho pan'āyasmante “parisuddhā”ti vedissāmi.

Yathā kho pana paccekapuṭṭhassa veyyākaraṇaṁ hoti, evamevaṁ evarūpāya parisāya yāvatatiyaṁ anusāvitaṁ hoti. Yo pana bhikkhu yāvatatiyaṁ anusāviyamāne saramāno santiṁ āpattiṁ nāvikareyya, sampajānamusāvādassa hoti. Sampajānamusāvādo kho pan'āyasmanto antarāyiko dhammo vutto bhagavatā, tasmā saramānena bhikkhunā āpannena visuddhāpekkhena santī āpatti āvikātabbā, āvikatā hissa phāsu hoti.

\medskip

\begin{center}
Uddiṭṭhaṁ kho āyasmanto nidānaṁ.

\smallskip

Tatth'āyasmante pucchāmi: Kacci'ttha parisuddhā?\\
Dutiyam'pi pucchāmi: Kacci'ttha parisuddhā?\\
Tatiyam'pi pucchāmi: Kacci'ttha parisuddhā?

\smallskip

Parisuddh'etth'āyasmanto, tasmā tuṇhī, evam'etaṁ dhārayāmi.
\end{center}

\begin{outro}
  nidān'uddeso paṭhamo
\end{outro}

\clearpage
