\setsecheadstyle{\sectionFmt}
\section{Saṅghādises'uddeso}
\label{sd}

\begin{intro}
  Ime kho pan'āyasmanto terasa saṅghādisesā dhammā uddesaṁ āgacchanti.
\end{intro}

\pdfbookmark[2]{Saṅghādisesa 1}{sd1}
\subsection*{\hyperref[comm1]{Saṅghādisesa 1: sukkavisaṭṭhisikkhāpadaṁ}}
\label{sd1}
Sañcetanikā sukkavisaṭṭhi aññatra supinantā saṅghādiseso.

\pdfbookmark[2]{Saṅghādisesa 2}{sd2}
\subsection*{\hyperref[comm2]{Saṅghādisesa 2: Kāyasaṁsaggasikkhāpadaṁ}}
\label{sd2}
Yo pana bhikkhu otiṇṇo vipariṇatena cittena mātugāmena saddhiṁ kāyasaṁsaggaṁ samāpajjeyya hatthagāhaṁ vā veṇigāhaṁ vā aññatarassa vā aññatarassa vā aṅgassa parāmasanaṁ, saṅghādiseso.

\pdfbookmark[2]{Saṅghādisesa 3}{sd3}
\subsection*{\hyperref[comm3]{Saṅghādisesa 3: Duṭṭhullavācāsikkhāpadaṁ}}
\label{sd3}
Yo pana bhikkhu otiṇṇo vipariṇatena cittena mātugāmaṁ duṭṭhullāhi vācāhi obhāseyya yathā taṁ yuvā yuvatiṁ methun'ūpasaṁhitāhi, saṅghādiseso.

\pdfbookmark[2]{Saṅghādisesa 4}{sd4}
\subsection*{\hyperref[comm4]{Saṅghādisesa 4: Attakāmapāricariyasikkhāpadaṁ}}
\label{sd4}
Yo pana bhikkhu otiṇṇo vipariṇatena cittena mātugāmassa santike attakāmapāricariyāya vaṇṇaṁ bhāseyya “etad'aggaṁ bhagini pāricariyānaṁ yā mādisaṁ sīlavantaṁ kalyāṇadhammaṁ brahmacāriṁ etena dhammena paricareyyā”ti methun'ūpasaṁhitena, saṅghādiseso.

\pdfbookmark[2]{Saṅghādisesa 5}{sd5}
\subsection*{\hyperref[comm5]{Saṅghādisesa 5: Sañcarittasikkhāpadaṁ}}
\label{sd5}
Yo pana bhikkhu sañcarittaṁ samāpajjeyya itthiyā vā purisamatiṁ purisassa vā itthimatiṁ, jāyattane vā jārattane vā, antamaso taṅkhaṇikāya'pi, saṅghādiseso.

\pdfbookmark[2]{Saṅghādisesa 6}{sd6}
\subsection*{\hyperref[comm6]{Saṅghādisesa 6: Kuṭikārasikkhāpadaṁ}}
\label{sd6}
Saññācikāya pana bhikkhunā kuṭiṁ kārayamānena assāmikaṁ att'uddesaṁ pamāṇikā kāretabbā, tatridaṁ pamāṇaṁ, dīghaso dvādasa vidatthiyo sugatavidatthiyā, tiriyaṁ sattantarā, bhikkhū abhinetabbā vatthudesanāya, tehi bhikkhūhi vatthuṁ desetabbaṁ anārambhaṁ saparikkamanaṁ. Sārambhe ce bhikkhu vatthusmiṁ aparikkamane saññācikāya kuṭiṁ kāreyya, bhikkhū vā anabhineyya vatthudesanāya, pamāṇaṁ vā atikkāmeyya, saṅghādiseso.

\pdfbookmark[2]{Saṅghādisesa 7}{sd7}
\subsection*{\hyperref[comm7]{Saṅghādisesa 7: Vihārakārasikkhāpadaṁ}}
\label{sd7}
Mahallakaṁ pana bhikkhunā vihāraṁ kārayamānena sassāmikaṁ att'uddesaṁ bhikkhū abhinetabbā vatthudesanāya, tehi bhikkhūhi vatthuṁ desetabbaṁ anārambhaṁ saparikkamanaṁ. Sārambhe ce bhikkhu vatthusmiṁ aparikkamane mahallakaṁ vihāraṁ kāreyya, bhikkhū vā anabhineyya vatthudesanāya, saṅghādiseso.

\pdfbookmark[2]{Saṅghādisesa 8}{sd8}
\subsection*{\hyperref[comm8]{Saṅghādisesa 8: Duṭṭhadosasikkhāpadaṁ}}
\label{sd8}
Yo pana bhikkhu bhikkhuṁ duṭṭho doso appatīto amūlakena pārājikena dhammena anuddhaṁseyya “app'eva nāma naṁ imamhā brahmacariyā cāveyya”nti, tato aparena samayena samanuggāhiyamāno vā asamanuggāhiyamāno vā amūlakañ'c'eva taṁ adhikaraṇaṁ hoti, bhikkhu ca dosaṁ patiṭṭhāti, saṅghādiseso.

\pdfbookmark[2]{Saṅghādisesa 9}{sd9}
\subsection*{\hyperref[comm9]{Saṅghādisesa 9: Aññabhāgiyasikkhāpadaṁ}}
\label{sd9}
Yo pana bhikkhu bhikkhuṁ duṭṭho doso appatīto aññabhāgiyassa adhikaraṇassa kiñci desaṁ lesamattaṁ upādāya pārājikena dhammena anuddhaṁseyya “app'eva nāma naṁ imamhā brahmacariyā cāveyya”nti, tato aparena samayena samanuggāhiyamāno vā asamanuggāhiyamāno vā aññabhāgiyañ'c'eva taṁ adhikaraṇaṁ hoti koci deso lesamatto upādinno, bhikkhu ca dosaṁ patiṭṭhāti, saṅghādiseso.

\pdfbookmark[2]{Saṅghādisesa 10}{sd10}
\subsection*{\hyperref[comm10]{Saṅghādisesa 10: Saṅghabhedasikkhāpadaṁ}}
\label{sd10}
Yo pana bhikkhu samaggassa saṅghassa bhedāya parakkameyya, bhedanasaṁvattanikaṁ vā adhikaraṇaṁ samādāya paggayha tiṭṭheyya, so bhikkhu bhikkhūhi evam'assa vacanīyo “mā āyasmā samaggassa saṅghassa bhedāya parakkami, bhedanasaṁvattanikaṁ vā adhikaraṇaṁ samādāya paggayha aṭṭhāsi, samet'āyasmā saṅghena, samaggo hi saṅgho sammodamāno avivadamāno ek'uddeso phāsu viharatī”ti, evañ'ca so bhikkhu bhikkhūhi vuccamāno tath'eva paggaṇheyya, so bhikkhu bhikkhūhi yāvatatiyaṁ samanubhāsitabbo tassa paṭinissaggāya, yāvatatiyañ'ce samanubhāsiyamāno taṁ paṭinissajjeyya, icc'etaṁ kusalaṁ, no ce paṭinissajjeyya, saṅghādiseso.

\pdfbookmark[2]{Saṅghādisesa 11}{sd11}
\subsection*{\hyperref[comm11]{Saṅghādisesa 11: Bhed'ānuvattakasikkhāpadaṁ}}
\label{sd11}
Tass'eva kho pana bhikkhussa bhikkhū honti anuvattakā vaggavādakā eko vā dve vā tayo vā, te evaṁ vadeyyuṁ “mā āyasmanto etaṁ bhikkhuṁ kiñci avacuttha, dhammavādī c'eso bhikkhu, vinayavādī c'eso bhikkhu, amhākañ'c'eso bhikkhu chandañ'ca ruciñca ādāya voharati, jānāti, no bhāsati, amhākam'p'etaṁ khamatī”ti, te bhikkhū bhikkhūhi evam'assu vacanīyā “mā āyasmanto evaṁ avacuttha, na c'eso bhikkhu dhammavādī, na c'eso bhikkhu vinayavādī, mā āyasmantānam'pi saṅghabhedo ruccittha, samet'āyasmantānaṁ saṅghena, samaggo hi saṅgho sammodamāno avivadamāno ek'uddeso phāsu viharatī”ti, evañ'ca te bhikkhū bhikkhūhi vuccamānā tath'eva paggaṇheyyuṁ, te bhikkhū bhikkhūhi yāvatatiyaṁ samanubhāsitabbā tassa paṭinissaggāya, yāvatatiyañ'ce samanubhāsiyamānā taṁ paṭinissajjeyyuṁ, icc'etaṁ kusalaṁ, no ce paṭinissajjeyyuṁ, saṅghādiseso.

\pdfbookmark[2]{Saṅghādisesa 12}{sd12}
\subsection*{\hyperref[comm12]{Saṅghādisesa 12: Dubbacasikkhāpadaṁ}}
\label{sd12}
Bhikkhu pan'eva dubbacajātiko hoti uddesapariyāpannesu sikkhāpadesu bhikkhūhi sahadhammikaṁ vuccamāno attānaṁ avacanīyaṁ karoti “mā maṁ āyasmanto kiñci avacuttha kalyāṇaṁ vā pāpakaṁ vā, aham'p'āyasmante na kiñci vakkhāmi kalyāṇaṁ vā pāpakaṁ vā, viramath'āyasmanto mama vacanāyā”ti, so bhikkhu bhikkhūhi evam'assa vacanīyo “mā āyasmā attānaṁ avacanīyaṁ akāsi, vacanīyam'ev'āyasmā attānaṁ karotu, āyasmā'pi bhikkhū vadetu sahadhammena, bhikkhūpi āyasmantaṁ vakkhanti sahadhammena, evaṁ saṁvaddhā hi tassa bhagavato parisā yad'idaṁ aññam'aññavacanena aññam'aññavuṭṭhāpanenā”ti, evañ'ca so bhikkhu bhikkhūhi vuccamāno tath'eva paggaṇheyya, so bhikkhu bhikkhūhi yāvatatiyaṁ samanubhāsitabbo tassa paṭinissaggāya, yāvatatiyañ'ce samanubhāsiyamāno taṁ paṭinissajjeyya, icc'etaṁ kusalaṁ, no ce paṭinissajjeyya, saṅghādiseso.

\pdfbookmark[2]{Saṅghādisesa 13}{sd13}
\subsection*{\hyperref[comm13]{Saṅghādisesa 13: Kuladūsakasikkhāpadaṁ}}
\label{sd13}
Bhikkhu pan'eva aññataraṁ gāmaṁ vā nigamaṁ vā upanissāya viharati kuladūsako pāpasamācāro, tassa kho pāpakā samācārā dissanti c'eva suyyanti ca, kulāni ca tena duṭṭhāni dissanti c'eva suyyanti ca, so bhikkhu bhikkhūhi evam'assa vacanīyo “āyasmā kho kuladūsako pāpasamācāro, āyasmato kho pāpakā samācārā dissanti c'eva suyyanti ca, kulāni c'āyasmatā duṭṭhāni dissanti c'eva suyyanti ca, pakkamat'āyasmā imamhā āvāsā, alan'te idha vāsenā”ti, evañ'ca so bhikkhu bhikkhūhi vuccamāno te bhikkhū evaṁ vadeyya “chandagāmino ca bhikkhū, dosagāmino ca bhikkhū, mohagāmino ca bhikkhū, bhayagāmino ca bhikkhū tādisikāya āpattiyā ekaccaṁ pabbājenti, ekaccaṁ na pabbājentī”ti, so bhikkhu bhikkhūhi evam'assa vacanīyo “mā āyasmā evaṁ avaca, na ca bhikkhū chandagāmino, na ca bhikkhū dosagāmino, na ca bhikkhū mohagāmino, na ca bhikkhū bhayagāmino, āyasmā kho kuladūsako pāpasamācāro, āyasmato kho pāpakā samācārā dissanti c'eva suyyanti ca, kulāni c'āyasmatā duṭṭhāni dissanti c'eva suyyanti ca, pakkamat'āyasmā imamhā āvāsā, alan'te idha vāsenā”ti, evañ'ca so bhikkhu bhikkhūhi vuccamāno tath'eva paggaṇheyya, so bhikkhu bhikkhūhi yāvatatiyaṁ samanubhāsitabbo tassa paṭinissaggāya, yāvatatiyañ'ce samanubhāsiyamāno taṁ paṭinissajjeyya, icc'etaṁ kusalaṁ, no ce paṭinissajjeyya, saṅghādiseso.

\medskip

\begin{center}
Uddiṭṭhā kho āyasmanto terasa saṅghādisesā dhammā nava paṭhamāpattikā, cattāro yāvatatiyakā. Yesaṁ bhikkhu aññataraṁ vā aññataraṁ vā āpajjitvā yāvat'ihaṁ jānaṁ paṭicchādeti, tāvat'ihaṁṁ tena bhikkhunā akāmā parivatthabbaṁ. Parivutthaparivāsena bhikkhunā uttariṁ chārattaṁ bhikkhumānattāya paṭipajjitabbaṁ, ciṇṇamānatto bhikkhu yattha siyā vīsatigaṇo bhikkhusaṅgho, tattha so bhikkhu abbhetabbo. Ekena'pi ce ūno vīsatigaṇo bhikkhusaṅgho taṁ bhikkhuṁ abbheyya, so ca bhikkhu anabbhito, te ca bhikkhū gārayhā, ayaṁ tattha sāmīci.

\smallskip

Tatth'āyasmante pucchāmi: Kacci'ttha parisuddhā?\\
Dutiyam'pi pucchāmi: Kacci'ttha parisuddhā?\\
Tatiyam'pi pucchāmi: Kacci'ttha parisuddhā?

\smallskip

Parisuddh'etth'āyasmanto, tasmā tuṇhī, evam'etaṁ dhārayāmī.
\end{center}

\begin{outro}
  saṅghādises'uddeso tatiyo
\end{outro}

\clearpage
