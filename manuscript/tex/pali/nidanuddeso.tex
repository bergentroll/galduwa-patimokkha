\section{Nidān'uddeso}
\label{nidan'uddeso}

Suṇātu me bhante / āvuso saṅgho, ajj'uposatho paṇṇaraso / cātuddaso / sāmaggo, yadi saṅghassa pattakallaṁ, saṅgho uposathaṁ kareyya pātimokkhaṁ uddiseyya.

Kiṁ saṅghassa pubbakiccaṁ?

Pārisuddhiṁ āyasmanto ārocetha. Pātimokkhaṁ uddisissāmi. Taṁ sabb'eva santā sādhukaṁ suṇoma manasikaroma. Yassa siyā āpatti, so āvikareyya. Asantiyā āpattiyā, tuṇhī bhavitabbaṁ. Tuṇhībhāvena kho pan'āyasmante parisuddhā'ti vedissāmi.

Yathā kho pana paccekapuṭṭhassa veyyākaraṇaṁ hoti, evam'evaṁ evarūpāya parisāya yāvatatiyaṁ anussāvitaṁ hoti. Yo pana bhikkhu yāvatatiyaṁ anussāviyamāne saramāno santiṁ āpattiṁ nāvikareyya, sampajānamusāvād'assa hoti. Sampajānamusāvādo kho pan'āyasmanto antarāyiko dhammo vutto bhagavatā. Tasmā saramānena bhikkhunā āpannena visuddh'āpekkhena santī āpatti āvikātabbā, āvikatā hi'ssa phāsu hoti.

\medskip

\linkdest{endnote8-body}
\begin{center}
Uddiṭṭhaṁ kho āyasmanto nidānaṁ.\makeatletter\hyperlink{endnote8-appendix}\Hy@raisedlink{\hypertarget{endnote8-body}{}{\pagenote{%
  \hypertarget{endnote8-appendix}{\hyperlink{endnote8-body}{This can be skipped since it doesn't occur in the Canon. The Nidāna can instead be concluded with ``Nidānaṁ niṭṭhitaṁ.''}}}}}\makeatother

\smallskip

Tatth'āyasmante pucchāmi: Kacci'ttha parisuddhā?\\
Dutiyam'pi pucchāmi: Kacci'ttha parisuddhā?\\
Tatiyam'pi pucchāmi: Kacci'ttha parisuddhā?

\smallskip

Parisuddh'etth'āyasmanto, tasmā tuṇhī, evam'etaṁ dhārayāmi.
\end{center}

\linkdest{endnote9-body}
\begin{outro}
  nidān'uddeso paṭhamo\makeatletter\hyperlink{endnote9-appendix}\Hy@raisedlink{\hypertarget{endnote9-body}{}{\pagenote{%
    \hypertarget{endnote9-appendix}{\hyperlink{endnote9-body}{Not in any edition or manuscript, but if a conclusion is to be recited then this one as given in the Parivāra would be the suitable one.\\
	When reciting in brief use: Nidān'uddeso niṭṭhito.}}}}}\makeatother
\end{outro}

\clearpage
